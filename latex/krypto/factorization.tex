\documentclass[a4paper,titlepage]{article}
\usepackage[english]{babel}
\usepackage[T1]{fontenc}
\usepackage[utf8]{inputenc}
\usepackage[pdftex]{graphics}
\usepackage{hyperref}
\usepackage{subfig}
\usepackage{graphicx}
\usepackage{enumerate}
\usepackage{amsmath}

\title{Factorization Project -- EDIN01}
\author{    Mats Rydberg, dt08mr7       \\
            Christina Schmidt, dt08cs6
  }
  
\begin{document}
\maketitle
\section*{Exercise 1}
We have a computational power $C = 10^6$ operations per second and we wish to naively try to factor a number $N$ of order $10^{25}$. This is done by performing the operation $N$ mod $p$ order of $\sqrt{N}$ number of times. The time $t$ this will take can be calculated as

$$
t = \frac{\sqrt{N}}{C} \approx \frac{10^{12}}{10^6} = 10^6 \text{ s} = 11 \text{ days } 13 \text{ h } 46 \text{ min and  } 40 \text{ s}
$$

This is of course not really feasible.

\section*{Exercise 2}
In this task we implement a simplified version of Quadratic Sieve, following the guidelines in the project description. The number $N$ that we will try to factor is given as

$$
N = 106 565 238 310 234 107 615 313 > 10^{24}
$$

\subsection*{Program}
The program is written in Java and is made up by four classes:
\begin{itemize}
\item \texttt{Main.java} which contains our main method and interacts with the user.
\item \texttt{Factorization.java} which includes the basic methods for doing actual factoring of numbers.
\item \texttt{Matrix.java} which is a wrapper for a primitive Java matrix and contains functionality for creating one that suits our needs.
\item \texttt{XandY.java} which computes the values $x$ and $y$ such that $x^2 = y^2$~mod~$N$ after the gaussian elimination step has been completed.
\end{itemize}

The program uses the \texttt{GaussBin} program provided for conducting the gaussian elimination step, so we make use of three text files: \texttt{primes.file},  \texttt{matrix.out} and \texttt{gauss.out}. The first contains the first $\sim 2000$ primes from which we read the $|F|$ primes used for our factor base, the second is our matrix written to the format specified as input for \texttt{GaussBin} and the third is the output from \texttt{GaussBin}, used as input for our final step in the algorithm.

\subsection*{Solution}
Our program solves the factoring of $N = p\cdot q$ in less than $780$ seconds $= 13$ minutes, as $p = $ and $q = $ on a powerful PC\footnote{Intel i5-2500K, 16GB RAM, Windows 7}, and does not finish in feasible time on a school computer\footnote{AMD Athlon II X2 B26 3.2GHz, 3.45GB RAM, Linux Mint}.

\subsection*{Extra metrics}
For extra goodies we provide a few extra metrics that we collected in the process of trying to optimize our program. They will make sense only in the context of the program itself.

\begin{verbatim}
text
\end{verbatim}

\vspace*{2cm}
\paragraph{Time spent on the project: 11 hours per person = 22 hours total}


\end{document}